\chapter{Test guide}

\prob{
    Solve the guide.
}

\newcommand{\frameme}[1]{\framebox[1.1\width]{#1}}
\begin{proof}
    The answers are inside a rectangle.
    \begin{itemize}
        \item[a] Hall's theorem and Frobenius' theorem
                    \begin{itemize}
                        \item[i.] Are not related.
                        \item[ii.] \frameme{Are equivalent.}
                        \item[iii.] One of them implies the other one, but the converse is not true.
                    \end{itemize}

        \item[b] The vertices of the polyhedron of matrices doubly stochastic are
                    \begin{itemize}
                        \item[i.] Orthogonal matrices.
                        \item[ii.] The non-singular 0-1 matrices.
                        \item[iii.] \frameme{Permutation matrices.}
                    \end{itemize}

        \item[c] A point $x_0$ of a polyhedron P is a vertex if
                    \begin{itemize}
                        \item[i.] \frameme{There exists a lineal function that is optimal (maximal or minimal) at $x_0$.}
                        \item[ii.] Every lineal function is minimized at $x_0$.
                        \item[iii.] No lineal function is minimzed at $x_0$.
                    \end{itemize}

        \item[d] If the factible region of a linear programming problem is empty then
                    \begin{itemize}
                        \item[i.] The factible region of the dual problem is also empty.
                        \item[ii.] The dual problem has an optimal solution.
                        \item[iii.] \frameme{The dual problem is unbounded.}
                    \end{itemize}

        \item[e] A matroid generalized the idea of
                    \begin{itemize}
                        \item[i.] Convexity.
                        \item[ii.] Polyhedrons.
                        \item[iii.] \frameme{Independence.}
                    \end{itemize}

        \item[f]
                    \begin{itemize}
                        \item[i.]
                        \item[ii.]
                        \item[iii.]
                    \end{itemize}

        \item[g]
                    \begin{itemize}
                        \item[i.]
                        \item[ii.]
                        \item[iii.]
                    \end{itemize}

        \item[h]
                    \begin{itemize}
                        \item[i.]
                        \item[ii.]
                        \item[iii.]
                    \end{itemize}

        \item[i]
                    \begin{itemize}
                        \item[i.]
                        \item[ii.]
                        \item[iii.]
                    \end{itemize}

        \item[j]
                    \begin{itemize}
                        \item[i.]
                        \item[ii.]
                        \item[iii.]
                    \end{itemize}

        \item[k]
                    \begin{itemize}
                        \item[i.]
                        \item[ii.]
                        \item[iii.]
                    \end{itemize}

        \item[l]
                    \begin{itemize}
                        \item[i.]
                        \item[ii.]
                        \item[iii.]
                    \end{itemize}

        \item[m]
                    \begin{itemize}
                        \item[i.]
                        \item[ii.]
                        \item[iii.]
                    \end{itemize}

        \item[n]
                    \begin{itemize}
                        \item[i.]
                        \item[ii.]
                        \item[iii.]
                    \end{itemize}

        \item[o]
                    \begin{itemize}
                        \item[i.]
                        \item[ii.]
                        \item[iii.]
                    \end{itemize}

        \item[p]
                    \begin{itemize}
                        \item[i.]
                        \item[ii.]
                        \item[iii.]
                    \end{itemize}

        \item[q]
                    \begin{itemize}
                        \item[i.]
                        \item[ii.]
                        \item[iii.]
                    \end{itemize}

        \item[r]
                    \begin{itemize}
                        \item[i.]
                        \item[ii.]
                        \item[iii.]
                    \end{itemize}

        \item[s]
                    \begin{itemize}
                        \item[i.]
                        \item[ii.]
                        \item[iii.]
                    \end{itemize}

        \item[t]
                    \begin{itemize}
                        \item[i.]
                        \item[ii.]
                        \item[iii.]
                    \end{itemize}

        \item[u]
                    \begin{itemize}
                        \item[i.]
                        \item[ii.]
                        \item[iii.]
                    \end{itemize}


    \end{itemize}
\end{proof}