\chapter{Test guide}

\prob{
    Solve the guide.
}

\newcommand{\frameme}[1]{\framebox[1.1\width]{#1}}
\begin{proof}
    The answers are inside a rectangle.
    \begin{itemize}
        \item[a] Hall's theorem and Frobenius' theorem
                    \begin{itemize}
                        \item[i.] Are not related.
                        \item[ii.] \frameme{Are equivalent.}
                        \item[iii.] One of them implies the other one, but the converse is not true.
                    \end{itemize}

        \item[b] The vertices of the polyhedron of matrices doubly stochastic are
                    \begin{itemize}
                        \item[i.] Orthogonal matrices.
                        \item[ii.] The non-singular 0-1 matrices.
                        \item[iii.] \frameme{Permutation matrices.}
                    \end{itemize}

        \item[c] A point $x_0$ of a polyhedron P is a vertex if
                    \begin{itemize}
                        \item[i.] \frameme{There exists a lineal function that is optimal (maximal or minimal) at $x_0$.}
                        \item[ii.] Every lineal function is minimized at $x_0$.
                        \item[iii.] No lineal function is minimzed at $x_0$.
                    \end{itemize}

        \item[d] If the factible region of a linear programming problem is empty then
                    \begin{itemize}
                        \item[i.] The factible region of the dual problem is also empty.
                        \item[ii.] The dual problem has an optimal solution.
                        \item[iii.] \frameme{The dual problem is unbounded.}
                    \end{itemize}

        \item[e] A matroid generalized the idea of
                    \begin{itemize}
                        \item[i.] Convexity.
                        \item[ii.] Polyhedrons.
                        \item[iii.] \frameme{Independence.}
                    \end{itemize}

        \item[f] If $A$ is a graph's incidence matrix. Then, for every $B$ sqare submatrex of $A$, 
                 $\det(B)$ belongs to:
                    \begin{itemize}
                        \item[i.] $\{-1, 0, 1\}$.
                        \item[ii.] $\{-1, \frac{-1}{2}, 0, \frac{1}{2}, 1\}$.
                        \item[iii.] \frameme{$\{-2, -1, \frac{-1}{2}, 0, \frac{1}{2}, 1, 2\}$.}
                    \end{itemize}

        \item[g] A rank function $r$ of a matroid satisfies one of the following properties for every
                subsets $A, B$.
                    \begin{itemize}
                        \item[i.] $r(A) + r(B) = r(A \cup B) + r(A \cap B)$.
                        \item[ii.] $r(A) + r(B) \leq r(A \cup B) + r(A \cap B)$.
                        \item[iii.] \frameme{$r(A) + r(B) \geq r(A \cup B) + r(A \cap B)$.}
                    \end{itemize}

        \item[h] A linear programming problem of the form $\min cx, Ax = b, x \geq 0$ is degenerated if
                    \begin{itemize}
                        \item[i.] There is no basic soltion in which a basic variable is zero.
                        \item[ii.] \frameme{There is a basic factible solution in which a basic variable is zero.} 
                        \item[iii.] In a basic factible solution a non-basic variable is zero.
                    \end{itemize}

        \item[i] A graphic is bipartite if and only if
                    \begin{itemize}
                        \item[i.] It doesn't contain triangles.
                        \item[ii.] \frameme{It doesn't contain odd cycles.}
                        \item[iii.] It doesn't contain a complete graph with four vertices.
                    \end{itemize}

        \item[j] If the propositions $x \in A$, $z \in B$ can be verified in polinomial time.
                 And if there is a theorem that states: there exists $x \in A$ or there exists
                 $z \in B$, but not both ($\exists x \in A \Leftrightarrow \not\exists z \in B$).
                Then, to find $x \in A$ is a problem in the class:
                    \begin{itemize}
                        \item[i.] $P$.
                        \item[ii.] \frameme{$NP \cap Co-NP$.} 
                        \item[iii.] $NP-complete$.
                    \end{itemize}

        \item[k] A graph is a tree if and only if
                    \begin{itemize}
                        \item[i.] It doesn't contain cycles.
                        \item[ii.] Is connected.
                        \item[iii.] \frameme{Is connected and it doesn't contain cycles.}
                    \end{itemize}

        \item[l] If $G = (V, E)$ is a graph with $k$ connected components, the rank of $E$ in the
                 graphic matroid is
                    \begin{itemize}
                        \item[i.] $|V|$.
                        \item[ii.] $2k$.
                        \item[iii.] \frameme{$|V| - k$.}
                    \end{itemize}

        \item[m]
                 Let $G = (V, E)$ be a directed graph with two distinguished vertices $s$ and $t$. The
                 blocking cutter for the $s-t$-trayectories is
                    \begin{itemize}
                        \item[i.] The set of directed cycles that contain $s$ and $t$.
                        \item[ii.] The set of directed trees with root $s$.
                        \item[iii.] \frameme{the set of directed cuts that separate $s$ and $t$.}
                    \end{itemize}

        \item[n]
                    \begin{itemize}
                        \item[i.]
                        \item[ii.]
                        \item[iii.]
                    \end{itemize}

        \item[o]
                    \begin{itemize}
                        \item[i.]
                        \item[ii.]
                        \item[iii.]
                    \end{itemize}

        \item[p]
                    \begin{itemize}
                        \item[i.]
                        \item[ii.]
                        \item[iii.]
                    \end{itemize}

        \item[q]
                    \begin{itemize}
                        \item[i.]
                        \item[ii.]
                        \item[iii.]
                    \end{itemize}

        \item[r]
                    \begin{itemize}
                        \item[i.]
                        \item[ii.]
                        \item[iii.]
                    \end{itemize}

        \item[s]
                    \begin{itemize}
                        \item[i.]
                        \item[ii.]
                        \item[iii.]
                    \end{itemize}

        \item[t]
                    \begin{itemize}
                        \item[i.]
                        \item[ii.]
                        \item[iii.]
                    \end{itemize}

        \item[u]
                    \begin{itemize}
                        \item[i.]
                        \item[ii.]
                        \item[iii.]
                    \end{itemize}


    \end{itemize}
\end{proof}