\chapter{De Morgan's Laws. (Finite Dimention Vector Spaces Case)}

\prob{
    Let $V$ be a vector space over a field $\F$ of finite dimension. 
    Let $A, B \subset V$ be vector subspaces. Prove that
    
    \begin{align}
            (A \wedge B)^\perp  &=  \paren{A^\perp \vee B^\perp} \label{p:orthogonal_de_morgan_first}\\
            \intertext{\center{and}}
            (A \vee B)^\perp    &=  \paren{A^\perp \wedge B^\perp} \label{p:orthogonal_de_morgan_second}\\ 
    \end{align}
}

\begin{proof}
    We are going to start with \eqref{p:orthogonal_de_morgan_first}.\pn
    
    First, let see that $\paren{A^\perp \vee B^\perp} \subset (A \wedge B)^\perp$.\pn
    
    Let be $a' \in A^\perp$, $b' \in B^\perp$ and $\lambda_1,\lambda_2 \in \F$. So $\paren{\lambda_1 a' + \lambda_2 b'} \in \paren{A^\perp \vee B^\perp}$. 
    And let be $c \in \paren{A \cap B}$. We have that\pn
    
    \begin{align}
            c \cdot (\lambda_1 a' + \lambda_2 b')   &=  \lambda_1 c \cdot a' + \lambda_2 c \cdot b'     \\
                                                    &=  0   +   \lambda_2 c \cdot b'                    \\
                                                    &\comment{because $c \in A$ and $a' \in A^\perp$}   \\
                                                    &=  0   +   0                                       \\
                                                    &\comment{because $c \in B$ and $a' \in B^\perp$}.   \\
    \end{align}\pn
    
    So, $\paren{A^\perp \vee B^\perp} \subset (A \wedge B)^\perp$, as we wanted.\pn
    
    Before continuing, lets make clear that $A^\perp \subset (A \wedge B)^\perp$. $A^\perp = \{ v \in V : v \cdot x = 0 \;\;\;\; \forall x \in A \}$. 
    Particularly if $A' \subset A$ and $v \in A^\perp$ we have that $v \cdot x = 0$ for all $x \in A'$. Given that $\paren{A \wedge B} \subset A$, 
    we have that $A^\perp \subset (A \wedge B)^\perp$.\pn
        
    Now we want to show that $(A \wedge B)^\perp \subset A^\perp \vee B^\perp$. Suppose that $x \notin A^\perp \vee B^\perp$. Particularly, 
    $x \notin A^\perp$. But we showed before that $A^\perp \subset (A \cap B)^\perp$, so we conclude that $x \notin (A \cap B)^\perp$.\pn
    
    For \eqref{p:orthogonal_de_morgan_second} we are going to use the fact that if $C \subset V$ is a vector subspace then
    $(C^\perp)^\perp = C$ (this holds whenever $V$ is a closed vector space, in particular, when $V$ is finite dimensional).
    
    \begin{align}
        (A^\perp \wedge B^\perp)^\perp  &=  (A^\perp)^\perp \vee (B^\perp)^\perp                            \\
                                        &\comment{Here we used \eqref{p:orthogonal_de_morgan_first}}        \\
                                        &=  A \vee B                                                        \\
                                        &\comment{Because of the previously mentioned fact}.
    \end{align}\pn
    
    Using the mentioned fact again, we obtain
    \begin{align}
        A^\perp \wedge B^\perp      &=  \paren{\paren{A^\perp \wedge B^\perp}^\perp}^\perp    \\
                                    &=  \paren{A \vee B}.
    \end{align}\pn
    
    As we wished to prove.
\end{proof}