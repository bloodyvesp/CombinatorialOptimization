\chapter{De Morgan's Laws. (Finite Dimention Vector Spaces Case)}

\prob{
    Let $V$ be a vector space over a field $\F$ of finite dimension. 
    Let $A, B \subset V$ be vector subspaces. Prove that
    
    \begin{align}
            (A \wedge B)^\perp &=  A^\perp \vee B^\perp \label{p:orthogonal_de_morgan_first}\\
            \intertext{\center{and}}
            (A \vee B)^\perp &=  A^\perp \wedge B^\perp \label{p:orthogonal_de_morgan_second}\\ 
    \end{align}
}

\begin{proof}
    We are going to start with \eqref{p:orthogonal_de_morgan_first}.\pn
    
    First, let see that $A^\perp \vee B^\perp \subset (A \wedge B)^\perp$. But, before, lets make clear that
    $A^\perp \subset (A \wedge B)^\perp$.\pn
    
    $A^\perp = \{ v \in V : v \cdot x = 0 \;\;\;\; \forall x \in A \}$. Particularly if $A' \subset A$ and $v \in A^\perp$
    we have that $v \cdot x = 0$ for all $x \in A'$. Given that $A \wedge B \subset A$, we have that $A^\perp \subset (A \wedge B)^\perp$.\pn
    
    The same argument proves that $B^\perp \subset (A \wedge B)^\perp$. Given that $A^\perp \vee B^\perp$ is a vector space,
    it contains every linear combination of the type $\lambda_1 a' + \lambda_2 b'$ for $\lambda_1, \lambda_2 \in \F$, $a' \in A^\perp$ and
    $b' \in B^\perp$. Which means $A^\perp \vee B^\perp \subset (A \wedge B)^\perp$.\pn
    
    Now we want to show that $(A \wedge B)^\perp \subset A^\perp \vee B^\perp$. Suppose that $x \notin A^\perp \vee B^\perp$. Particularly, 
    $x \notin A^\perp$. But we showed before that $A^\perp \subset (A \cap B)^\perp$, so we conclude that $x \notin (A \cap B)^\perp$.\pn
    
    For \eqref{p:orthogonal_de_morgan_second} we are going to use the fact that if $C \subset V$ is a vector subspace then
    $(C^\perp)^\perp = C$.
    
    \begin{align}
        (A^\perp \wedge B^\perp)^\perp  &=  (A^\perp)^\perp \vee (B^\perp)^\perp                            \\
                                        &\comment{Here we used \eqref{p:orthogonal_de_morgan_first}}        \\
                                        &=  A \vee B                                                        \\
                                        &\comment{Because of the previously mentioned fact}.
    \end{align}\pn
    
    Using the mentioned fact again, we obtain
    \begin{align}
        A^\perp \wedge B^\perp      &=  \paren{\paren{A^\perp \wedge B^\perp}^\perp}^\perp    \\
                                    &=  \paren{A \vee B}.
    \end{align}\pn
    
    As we wished to prove.
\end{proof}