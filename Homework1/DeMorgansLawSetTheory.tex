\chapter{De Morgan's Laws. (Set Theory Case)}

\prob{
    Given a set $\Omega$, and subsets $A, B \subset \Omega$, prove that
    \begin{align}
         (A \cap B)^c &=  A^c \cup B^c \label{p:de_morgan_first}\\
            \intertext{\center{and}}
         (A \cup B)^c &=  A^c \cap B^c \label{p:de_morgan_second}\\                     
    \end{align}
}

\begin{proof}
    We are going to start with \eqref{p:de_morgan_first}.\pn
    
    First, lets see that $(A \cap B)^c \subset  A^c \cup B^c$. So let $x \in (A \cap B)^c$. That means,
    $x \notin A \cap B$.\pn
    
    As $A \cap  B = \{ y \in \Omega : y \in A \text{ and } y \in B \}$ it follows that
    $x \notin A$ or $x \notin B$. That is $x \in A^c$ or $x \in B^c$, which is $x \in A^c \cup B^c$ and then
    we have $(A \cap B)^c \subset  A^c \cup B^c$.\pn
    
    Now, let $x \in A^c \cup B^c$.\pn 
    
    Suppose that $x \notin \paren{A \cap B}^c$. That is, $x \in A \cap B$. Then $x \in A$ and $x \in B$. So
    $x \notin A^c$ and $x \notin B^c$. Which contradicts that $x \in A^c \cup B^c$.\pn
    
    Now lets prove \eqref{p:de_morgan_second}. We are going to use the fact that for any $C \subset \Omega$,
    it happens that $(C^c)^c = C$.\pn
    
    \begin{align}
        \paren{A^c \cap B^c}^c  &=  (A^c)^c \cup (B^c)^c                                    \\
                                &\comment{Here we applied \eqref{p:de_morgan_first}}        \\
                                &=  A \cup B                                                \\
                                &\comment{Here we applied the fact previously mentioned}.   \\
    \end{align}\pn
    
    With this, and appling the mentioned fact one more time, we get
    
    \begin{align}
        \paren{A^c \cap B^c}    &=  \paren{\paren{A^c \cap B^c}^c}^c        \\
                                &=  \paren{A \cup B}^c.
    \end{align}
\end{proof}