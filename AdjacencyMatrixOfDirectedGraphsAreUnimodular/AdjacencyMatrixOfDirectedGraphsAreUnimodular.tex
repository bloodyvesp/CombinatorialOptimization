\chapter{Adjacency Matrix Of Directed Graphs Are Unimodular}

\prob{
    Let $G = (V, E)$ be a directed graph. And let $M$ be its adjacency matrix.
    Prove that if $A$ is a square submatrix of $M$ then $det(A) \in \{0, -1, 1\}$.
}

\begin{proof}
    The proof of this statement is completily analog to the one of [\ref{AdjacencyMatrixOfBipartiteGraphsAreUnimodular}].\pn
    
    Again, the proof will be by induction over the dimension $n$ of $A$.\pn
    
    If $n=1$, then the statement is straightforward true as the entries of an adjacency matrix of a directed graph are always
    $1$, $-1$ or $0$. Then its determintant will be $1$, $-1$ or $0$ respectivelly.\pn
    
    Now, suppose that the statement is true for $n \geq 1$ and let $A$ be a $(n+1) \times (n+1)$ submatrix of $M$.\pn
    
    If $A$ contains a column with only zeros, then its determinant will be $0$ and we are done.\pn
    
    If $A$ contains a column with only one $1$ or only one $-1$, and lets say that this $1$ (or $-1$) is in the entry $(i,j)$
    then $\det(A)$ will be $-\det(A_{i,j})$ or $\det(A_{i,j})$ (where $A_{i,j}$ is the submatrix obtained from deleting the 
    $i$-th row and $j$-th column of $A$). As $A_{i,j}$ is a submatrix of $M$ with dimension $n \times n$ our induction hypotesis holds
    and then $\det(A) \in \{1, -1, 0\}$.\pn
    
    If neither one of these cases occur, then $A$ contains exactly a $-1$ and a $1$ in each of its columns, then, if we
    sum all their rows, what we get is a zero $n$-tuple. In other words, there is a linear combination of the rows distinct from the
    zero linear combination that gets the zero vector, that is, the row vectors and linearly dependent and then $\det(A) = 0$.
\end{proof}