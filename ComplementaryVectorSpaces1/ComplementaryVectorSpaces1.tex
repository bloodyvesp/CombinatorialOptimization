\chapter{Complementary Vector Spaces (1)}
    The order of the following two excercises has been changed because the second one is
    a very particular case of this one, so it is convenient to solve this one first.

    \prob{
        Let $A$ be a $m \times n$ real matrix with linearly independent rows.
        Lets define $H = \{x \in \R^n | Ax^T = 0 \}$ and 
        $L = \{ y \in \R^n | y = \lambda A, \; \lambda \in \R^m\}$.\pn
        
        Prove that $H$ and $L$ are orthogonal complements of each other.
    }
    
    \begin{proof}
        What we need to proove is that $H = L^\perp$.\pn
        
        It is easy to see that $H \subset L^\perp$ given that if $x \in H$ and $y \in L$, then
        there is some $\lambda \in \R^m$ such that $y = \lambda A$ and
        
        \begin{align}
                yx^T    &=  (\lambda A) x^T \\
                        &=  \lambda (A x^T) \\
                        &=  \lambda (0^T)   \\
                        &=  0.
        \end{align}\pn
        
        Now lets suppose that $z \in L^\perp$, we would like to prove that $A z^T = 0$, and therefore,
        $z \in H$.\pn
        
        So, given that $z \in L^\perp$, then, for every $y \in L$ we have
        \begin{align}
                    yz^T = 0.
        \end{align}\pn
        
        But, every $y \in L$ can be written as $y = \lambda A$ with $\lambda \in \R^m$. So, 
        for every $\lambda \in \R^m$ we have 
        
        \begin{align}
                \lambda A z^T   = 0.
        \end{align}
        
        Now, $A z^T \in \R^m$, and it is such that it is orthogonal with every other vector 
        $\lambda \in \R^m$. And there is only one vector with such property and that is the
        zero vector of $\R^m$. Then, given that $A z^T = 0$, $z \in H$, as we wished to proove.
    \end{proof}