\chapter{Permutohedron Vertices As Polymatroid}

    \prob{
        The permutohedron of dimension $n-1$ is defined as follows.\pn
        
        Lets call $E = \{1, 2, \dots, n\}$. Lets be $(x_1, x_2, \dots, x_n) \in \R^n$ an entry in $\R^n$.
        For $S \subset E$, $x(S) = \sum_{i \in S} x_i$. The permutohedron of dimension $n-1$ is the polytope
        defined by the following linear bounds.
        \begin{align}
                x(S)    &= 1 + 2 + \dots + n                &\text{if \; $S = E$}\\
                x(S)    &\leq (k+1) + (k+2) + \dots + n     &\text{if $S \neq E$, where $ k = n - |S|$}\\
                x_i     &\geq 0                             &\text{for $1 \leq i \leq n$}.        
        \end{align}
        
        The polymatroid associated with a permutohedron is the one obtained from converting the first equality
        in a $\leq$ inequality. That is, the polytope given by
        \begin{align}
                x(S)    &\leq (k+1) + (k+2) + \dots + n     &\text{where $ k = n - |S|$}\\       
                x_i     &\geq 0                             &\text{for $1 \leq i \leq n$}.        
        \end{align}
        
        What are the vertices of this polymatroid and how are these associated with a matroid?
    }
    
    \begin{proof}
        Lets find the vertices. We are going to give the vertices by finding $n$ equalities given by the inequalities system.\pn
        
        First, notice that $(0, 0, \dots, 0)$ is a vertex, given that it satisfaces the $n$ equalities with the form 
        \begin{align}
                x_i     &= 0                                &\text{for $1 \leq i \leq n$}.
        \end{align}
        
        We are going to procede loosening one of this equalities, increasing the loose $x_i$ until it meets another equality. For example,
        we loosen the equality $x_1 = 0$, and we increase $x_1$ until it meets some other equality. The first equality that it meets is
        $x_i = n$ (which comes from the equality $x_i \leq n$). So now we have that the vertex $(n, 0, \dots, 0)$ is a vertex of our
        polymatroid.\pn
        
        Now, we loosen some other equality and increase the corresponding $x_i$ until it meets some other equality. From our example, lets say, 
        we loosen $x_2 = 0$, and we increase $x_2$ until it meets some other equality. The first equality that it meets is $x_1 + x_2 = (n-1) + n$
        (whice comes from the inequality $x_1 + x_2 \leq (n-1) + n$). So now he have that the vertex $(n, n-1, 0, \dots, 0)$ is a vertex of our
        polymatroid.\pn
        
        We proceed the same way with all the other $n-2$ equalities. Then we notice that the order we choose is irrelevant and in the $m$ step, we
        have a vertex with $m-1$ non-zero entries, and even more, the non-zero entries are exactly the $m-1$ largest values from $\{1, 2, \dots, n\}$.\pn
        
        So, now we know which are the vertices of our polymatroid and with this procedure, we can even count how many of them are.
        There is only one vertex with all of its entries equal to zero. Then, we have $n$, vertices with only one non-zero entry (and the value of its non-zero
        entry is exactly $n$). Then, we have $\binom{n}{2} 2!$ vertices with two non-zero entries (and the value of its non-zero entries are $n$ and $n-1$), 
        the $\binom{n}{2}$ counts the ways we can choose the non-zero entries, and the $2!$ counts the way we can asing their values. In the general case,
        we have $\binom{n}{m} m!$ vertices with exactly $m$ non-zero entries (and the valueof its non-zero entries are $n, n-1, \dots, n-m+1$).\pn
        
        So, in the end we have
        \begin{align}
                \sum_{m = 0}^n \binom{n}{m} m!
        \end{align}
        
        vertices for the polymatroid.\pn
        
        Other way to count them could be, counting what is the loose $x_i$ in each step. Lets say, in the first step we don't loosen any equality, so 
        we have only one vertex there. In the second step, we choose one of the $n$ zero-equalities to be loose and its corresponding variable will get the value $n$, 
        so there we have another $n = \frac{n!}{(n-1)!}$ vertices. In the third step, we choose one of the $n-1$ zero-equalities to be loose and its corresponding
        variable will get the value $n-1$, so there we have another $n (n-1) = \frac{n!}{(n-2)!}$ vertices. In the general step, we have to choose one of the $n-m$ 
        equalities to be loose and its corresponding variable will get the value $n-m$, and there we will get another $n (n-1) \cdots (n-m) = \frac{n!}{(n-m-1)!}$ and 
        this will repeat until $m = n-1$ (when we have to choose the only remaining zero-equality to be loosen and its corresponding variable will get the value $1$).\pn
        
        So, in the end we have
        \begin{align}
                \sum_{m = 0}^{n} \frac{n!}{(n-m)!}
        \end{align}
        
        Which, with a little algebra we can see that is exactly the same than before.\pn
        
        To see an association with a matroid, we need to develop a sense of independence that meets our expectations.\pn
        
        Lets say that a $n$-tuple is independent if its coordinates consist of zeros and a subset of the last and consecutive numbers from $\{1, 2 \dots, n\}$.
        And we do not worry about order they are given in. Lets say that a $n$-tuple $x$ contains another $n$-tuple $y$ if the difference $x-y$ doesn't contain
        negative numbers (so here, the order of the non-zero entries does matter).\pn
        
        Given this definitions, the zero vector is independent and it is contained in any other independent vertex. And, converting a $0$ from one of these
        vertices into the value of its less positive coordinet minus one, we will have a vertex that contains it (a super vertex). As we have only a finite
        number of coordinates, this will stop, and when that happens we will have a ``super super vertex'' or lets say, a basis.\pn
        
        We could define a ``size'' of a vertex as the number of non-zero coordinates. And this way we have satisfied almost all the matroids' hypotesis, basis
        are the independent vertex of maximum size, any other independent vertex is contained in a basis, if an independent vertex is contained in another one
        there is a coordinate that could be changed to obtain a new independent vertex. But, there is something that we will not be able to define, and this is
        the set operation ``union'', because, if two different vertex has its coordinate with value $n$ in different position any super vertex for these should have
        its $n$ valued coordinate in the same position, but it cannot have two of them.\pn
        
        At least, we were able to give a ``sense'' of matroid to this polimatroid.\pn        
    \end{proof}