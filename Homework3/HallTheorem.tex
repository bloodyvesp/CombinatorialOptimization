\chapter{Hall's Theorem}

    \begin{theorem}
        Let $G = (V_1 \cup V_2, E)$ be a finite bipartite graph such that $V_1 \cap V_2 = \emptyset$ and $\left|V_1\right| = \left|V_2\right|$.
        There is a perfect matching if and only if for every $A \subset V_1$, it happens that $\left|A\right| \leq \left|N_G(A)\right|$. Where
        $N_G(A)$ means ``The neighbors of $A$ in $G$''.
    \end{theorem}
    
    \begin{proof}
        Lets remember what Frobenius' theorem says. It says that if $M$ is an $n \times n$ matrix with only $ones$ and $zeros$, then neither,
        $M$ contains a permutation matrix (other way to say it is that there exists a permutation $\sigma \in S_n$ such that $M_{i, \sigma(i)} = 1$ 
        for each $i$) or it has a $s \times t$ zero submatrix such that $s + t > n$.\pn
        
        Now, lets find the connection between these two theorems. We can represent a bipartite graph such as the one mentioned in Hall's theorem by 
        a $n \times n$ matrix $M$, in which each row will represent a vertex in $V_1$ and each column a vertex in $V_2$, and, if vertex $i \in V_1$ and 
        $j \in V_2$ are neighbors, then the entry $M_{i,j}$ will be $1$, and $0$ otherwise.\pn
        
        Given this representation, a matching in $G$ will look as a matrix in which each column and each row have exactly one $1$, and that is,
        a permutation matrix. So Frobenius' theorem will give us the answer.\pn
        
        If $G$ doesn't contain a matching, that is, $M$ doesn't contain a permutation matrix. Then, Frobenius' theorem says that there is 
        a $s \times t$ zero submatrix such that $s + t > n$. Now lets think in the $s$ rows of $M$ that will be part of such zero submatrix.
        This $s$ rows will have only $0$'s in the $t$ columns that correspond to the columns of the zero submatrix. This means that the $s$ vertex that
        those rows represent only can have $n - t$ neighbors. But the inequallity $s + t > n$, also says that $s > n - t$. So we have a
        set of $s$ vertices such that the amount of their neighbors is less than $s$. And with this we have proved Hall's theorem.        
    \end{proof}