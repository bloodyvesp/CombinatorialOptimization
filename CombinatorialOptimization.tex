%% Based on a TeXnicCenter-Template by Gyorgy SZEIDL.
%%%%%%%%%%%%%%%%%%%%%%%%%%%%%%%%%%%%%%%%%%%%%%%%%%%%%%%%%%%%%

%------------------------------------------------------------
%
\documentclass{amsart}
%
%----------------------------------------------------------
% This is a sample document for the AMS LaTeX Article Class
% Class options
%        -- Point size:  8pt, 9pt, 10pt (default), 11pt, 12pt
%        -- Paper size:  letterpaper(default), a4paper
%        -- Orientation: portrait(default), landscape
%        -- Print size:  oneside, twoside(default)
%        -- Quality:     final(default), draft
%        -- Title page:  notitlepage, titlepage(default)
%        -- Start chapter on left:
%                        openright(default), openany
%        -- Columns:     onecolumn(default), twocolumn
%        -- Omit extra math features:
%                        nomath
%        -- AMSfonts:    noamsfonts
%        -- PSAMSFonts  (fewer AMSfonts sizes):
%                        psamsfonts
%        -- Equation numbering:
%                        leqno(default), reqno (equation numbers are on the right side)
%        -- Equation centering:
%                        centertags(default), tbtags
%        -- Displayed equations (centered is the default):
%                        fleqn (equations start at the same distance from the right side)
%        -- Electronic journal:
%                        e-only
%------------------------------------------------------------
% For instance the command
%          \documentclass[a4paper,12pt,reqno]{amsart}
% ensures that the paper size is a4, fonts are typeset at the size 12p
% and the equation numbers are on the right side
%
\usepackage{amsmath}%
\usepackage{amsfonts}%
\usepackage{amssymb}%
\usepackage{graphicx}
%------------------------------------------------------------
% Theorem like environments
%
\newtheorem{theorem}{Theorem}
\theoremstyle{plain}
\newtheorem{acknowledgement}{Acknowledgement}
\newtheorem{algorithm}{Algorithm}
\newtheorem{axiom}{Axiom}
\newtheorem{case}{Case}
\newtheorem{claim}{Claim}
\newtheorem{conclusion}{Conclusion}
\newtheorem{condition}{Condition}
\newtheorem{conjecture}{Conjecture}
\newtheorem{corollary}{Corollary}
\newtheorem{criterion}{Criterion}
\newtheorem{definition}{Definition}
\newtheorem{example}{Example}
\newtheorem{exercise}{Exercise}
\newtheorem{lemma}{Lemma}
\newtheorem{notation}{Notation}
\newtheorem{problem}{Problem}
\newtheorem{proposition}{Proposition}
\newtheorem{remark}{Remark}
\newtheorem{solution}{Solution}
\newtheorem{summary}{Summary}
\numberwithin{equation}{section}
%--------------------------------------------------------
\begin{document}
\title[Short Title (for the running head)]{Full Title}
\author{Author One}
\address[A. One and A. Two]
{Author OneTwo common address, line 1 \newline%
\indent Author OneTwo common address, line 2}%
\email[A. One]{author-one@autherone-inst.de}%
\urladdr{http://www.authorone.uni-aone.de}
\author{Author Two}
\curraddr[A.~Two]{Author Two current address, line 1\newline%
\indent Author Two current address, line 2}%
\email[A.~Two]{author-two@authortwo-inst.hu}%
\urladdr{http://www.authortwo.uni-atwo.hu}
\author{Author Three}
\address[A. Three]{Author Three address, line 1\newline%
\indent Author Three address, line 2}
\email[A.~Three]{author-three@authorthree-inst.edu}%
\urladdr{http://www.authorthree.uni-athree.edu}
\thanks{Thanks for Author One.}
\thanks{Thanks for Author Two.}
\thanks{This paper is in final form and no version of it will be submitted for
publication elsewhere.}
\date{March 15, 2002}
\subjclass{Primary 05C38, 15A15; Secondary 05A15, 15A18} %
\keywords{Keyword one, keyword two etc.}%
\dedicatory{Dedicated to Professor XY on the occasion of his seventieth birthday.}

\begin{abstract}
This is a sample document which shows the most important features of the AMS Journal
Article class.
\end{abstract}
\maketitle


\section{Introduction}

\noindent The front matter has various entries such as\\
\hspace*{\fill}\verb" \title", \verb"\author", \verb"\address", \verb"\e-mail",
\verb"\urladdress" etc. \hspace*{\fill}\\
You should replace their arguments with your own.

This text is the body of your article. You may delete everything between the commands\\
\hspace*{\fill} \verb"\begin{document}" \ldots \verb"\end{document}"
\hspace*{\fill}\\in this file to start with a blank document.

\section{The Most Important Features}

\noindent Sectioning commands. The first one is the\\
\hspace*{\fill} \verb"\section{The Most Important Features}" \hspace*{\fill}\\
command. Below you shall find examples for further sectioning commands:

\subsection{Subsection}
Subsection text.

\subsubsection{Subsubsection}
Subsubsection text.

\paragraph{Paragraph}
Paragraph text.

\subparagraph{Subparagraph}Subparagraph text.\vspace{2mm}

Select a part of the text then click on the button Emphasize (H!), or
Bold (Fs), or Italic (Kt), or Slanted (Kt) to typeset \emph{Emphasize},
\textbf{Bold}, \textit{Italics}, \textsl{Slanted} texts.

You can also typeset \textrm{Roman}, \textsf{Sans Serif}, \textsc{Small Caps},
and \texttt{Typewriter} texts.

You can also apply the special, mathematics only commands $\mathbb{BLACKBOARD}$
$\mathbb{BOLD}$, $\mathcal{CALLIGRAPHIC}$, and $\mathfrak{fraktur}$. Note that
blackboard bold and calligraphic are correct only when applied to uppercase
letters A through Z.

You can apply the size tags -- Format menu, Font size submenu -- {\tiny tiny},
{\scriptsize scriptsize}, {\footnotesize footnotesize}, {\small small},
{\normalsize normalsize}, {\large large}, {\Large Large}, {\LARGE LARGE},
{\huge huge} and {\Huge Huge}.

You can use the \verb"\begin{quote} etc. \end{quote}" environment for typesetting
short quotations. Select the text then click on Insert, Quotations, Short Quotations:

\begin{quote}
The buck stops here. \emph{Harry Truman}

Ask not what your country can do for you; ask what you can do for your
country. \emph{John F Kennedy}

I am not a crook. \emph{Richard Nixon}

I did not have sexual relations with that woman, Miss Lewinsky. \emph{Bill Clinton}
\end{quote}

The Quotation environment is used for quotations of more than one paragraph. Following
is the beginning of \emph{The Jungle Books} by Rudyard Kipling. (You should select
the text first then click on Insert, Quotations, Quotation):

\begin{quotation}
It was seven o'clock of a very warm evening in the Seeonee Hills when Father Wolf woke
up from his day's rest, scratched himself, yawned  and spread out his paws one after
the other to get rid of sleepy feeling in their tips. Mother Wolf lay with her big gray
nose dropped across her four tumbling, squealing cubs, and the moon shone into the
mouth of the cave where they all lived. ``\emph{Augrh}'' said Father Wolf, ``it is time
to hunt again.'' And he was going to spring down hill when a little shadow with a bushy
tail crossed the threshold and whined: ``Good luck go with you, O Chief of the Wolves;
and good luck and strong white teeth go with the noble children, that they may never
forget the hungry in this world.''

It was the jackal---Tabaqui the Dish-licker---and the wolves of India despise Tabaqui
because he runs about making mischief, and telling tales, and eating rags and pieces of
leather from the village rubbish-heaps. But they are afraid of him too, because
Tabaqui, more than any one else in the jungle, is apt to go mad, and then he forgets
that he was afraid of anyone, and runs through the forest biting everything in his way.
\end{quotation}

Use the Verbatim environment if you want \LaTeX\ to preserve spacing, perhaps when
including a fragment from a program such as:
\begin{verbatim}
#include <iostream>         // < > is used for standard libraries.
void main(void)             // ''main'' method always called first.
{
 cout << ''This is a message.'';
                            // Send to output stream.
}
\end{verbatim}
(After selecting the text click on Insert, Code Environments, Code.)


\subsection{Mathematics and Text}

It holds \cite{KarelRektorys} the following
\begin{theorem}
(The Currant minimax principle.) Let $T$ be completely continuous selfadjoint operator
in a Hilbert space $H$. Let $n$ be an arbitrary integer and let $u_1,\ldots,u_{n-1}$ be
an arbitrary system of $n-1$ linearly independent elements of $H$. Denote
\begin{equation}
\max_{\substack{v\in H, v\neq
0\\(v,u_1)=0,\ldots,(v,u_n)=0}}\frac{(Tv,v)}{(v,v)}=m(u_1,\ldots, u_{n-1}) \label{eqn10}
\end{equation}
Then the $n$-th eigenvalue of $T$ is equal to the minimum of these maxima, when minimizing over all linearly independent systems $u_1,\ldots u_{n-1}$ in $H$,
\begin{equation}
\mu_n = \min_{\substack{u_1,\ldots, u_{n-1}\in H}} m(u_1,\ldots, u_{n-1}) \label{eqn20}
\end{equation}
\end{theorem}
The above equations are automatically numbered as equation (\ref{eqn10}) and
(\ref{eqn20}).

\subsection{List Environments}

You can create numbered, bulleted, and description lists
(Use the Itemization or Enumeration buttons, or click on the Insert menu
then chose an item from the Enumeration submenu):

\begin{enumerate}
\item List item 1

\item List item 2

\begin{enumerate}
\item A list item under a list item.

However, the typeset style for this level is different.

\item Just another list item under a list item.

\begin{enumerate}
\item Third level list item under a list item.

\begin{enumerate}
\item Fourth and final level of list items allowed.
\end{enumerate}
\end{enumerate}
\end{enumerate}
\end{enumerate}

\begin{itemize}
\item Bullet item 1

\item Bullet item 2

\begin{itemize}
\item Second level bullet item.

\begin{itemize}
\item Third level bullet item.

\begin{itemize}
\item Fourth (and final) level bullet item.
\end{itemize}
\end{itemize}
\end{itemize}
\end{itemize}

\begin{description}
\item[Description List] Each description list item has a term followed by the
description of that term. Double click the term box to enter the term, or to
change it.

\item[Bunyip] Mythical beast of Australian Aboriginal legends.
\end{description}

\subsection{Theorem-like Environments}

The following theorem-like environments (in alphabetical order) are available
in this style.

\begin{acknowledgement}
This is an acknowledgement
\end{acknowledgement}

\begin{algorithm}
This is an algorithm
\end{algorithm}

\begin{axiom}
This is an axiom
\end{axiom}

\begin{case}
This is a case
\end{case}

\begin{claim}
This is a claim
\end{claim}

\begin{conclusion}
This is a conclusion
\end{conclusion}

\begin{condition}
This is a condition
\end{condition}

\begin{conjecture}
This is a conjecture
\end{conjecture}

\begin{corollary}
This is a corollary
\end{corollary}

\begin{criterion}
This is a criterion
\end{criterion}

\begin{definition}
This is a definition
\end{definition}

\begin{example}
This is an example
\end{example}

\begin{exercise}
This is an exercise
\end{exercise}

\begin{lemma}
This is a lemma
\end{lemma}

\begin{proof}
This is the proof of the lemma.
\end{proof}

\begin{notation}
This is notation
\end{notation}

\begin{problem}
This is a problem
\end{problem}

\begin{proposition}
This is a proposition
\end{proposition}

\begin{remark}
This is a remark
\end{remark}

\begin{solution}
This is a solution
\end{solution}

\begin{summary}
This is a summary
\end{summary}

\begin{theorem}
This is a theorem
\end{theorem}

\begin{proof}
[Proof of the Main Theorem]This is the proof.
\end{proof}

This text is a sample for a short bibliography. You can cite a book by making use of
the command \verb"\cite{KarelRektorys}": \cite{KarelRektorys}. Papers can be cited
similarly: \cite{Bertoti97}. If you want multiple citations to appear in a single set
of square brackets you must type all of the citation keys inside a single citation,
separating each with a comma. Here is an example: \cite{Bertoti97, Szeidl2001,
Carlson67}.

\begin{thebibliography}{9}                                                                                                %
\bibitem {KarelRektorys}Rektorys, K., \textit{Variational methods in Mathematics,
Science and Engineering}, D. Reidel Publishing Company,
Dordrecht-Hollanf/Boston-U.S.A., 2th edition, 1975

\bibitem {Bertoti97} \textsc{Bert\'{o}ti, E.}:\ \textit{On mixed variational formulation
of linear elasticity using nonsymmetric stresses and displacements}, International
Journal for Numerical Methods in Engineering., \textbf{42}, (1997), 561-578.

\bibitem {Szeidl2001} \textsc{Szeidl, G.}:\ \textit{Boundary integral equations for
plane problems in terms of stress functions of order one}, Journal of Computational and
Applied Mechanics, \textbf{2}(2), (2001), 237-261.

\bibitem {Carlson67}  \textsc{Carlson D. E.}:\ \textit{On G\"{u}nther's stress functions
for couple stresses}, Quart. Appl. Math., \textbf{25}, (1967), 139-146.
\end{thebibliography}
\end{document}
