\chapter{Linear programing examples}

\prob{
    Solve the two examples given as homework.
}

\begin{proof}
    I must confess that for this excercices I was a little too lazy to solve them with my bare hands.
    When it comes to matrix operations there is always something that makes me commit mistakes.\pn
    
    I may be bad at doing the math for matrix operations, but my computer is very efficient doing so,
    and I and my computer get along  very well, so I programmed the simplex method to solve this problem.\pn
    
    Here you will find my implementation (which could not be elegant, but it works). You can also download my code from
		the git repository that I mentioned in the ``Note for the readers" section. The programming language is Octave, and it should
		run as well with its paid version MathLab.\pn
    
		\newpage
    This is my simplex algorithm implementation:
		\small
    \lstinputlisting{Simplex/simplex.m}
		\normalsize
		
		\newpage
    This is my simplex method implementation:
		\small
    \lstinputlisting{Simplex/simplex_method.m}
		\normalsize
		
		
		\newpage
		\section{example 1}
		In the next bunch of pages, you will see the algorithm output. It will seem a little too excesive. 
		There are a few steps that could be optimizated in order to have a smaller output but that will make
		my algorithm to be ``less general"". I'll try to improve the output only for readiblity purposes,
		but I cannot say that it will be ready soon.\pn
		
    This is the output for the example 1.
		\small
    \lstinputlisting{Simplex/example1.results.txt}
		\normalsize
		The last lines say that we reach the optimum value at \texttt{x1=2, x2=3, x3=1, x4=4} and that the optimum value is \texttt{34}.
		
		\newpage
		\section{example 2}
    This is the output for the example 2.
		\small
    \lstinputlisting{Simplex/example2.results.txt}
		\normalsize
		The last lines say that we reach the optimum value at \texttt{x1=2, x2=3, x3=1} and that the optimum value is \texttt{14}.
\end{proof}