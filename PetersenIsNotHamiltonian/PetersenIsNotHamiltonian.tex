\chapter{Petersen's Graph Is Not Hamiltonian}

    \prob{
        Prove that Petersen's graph is not hamiltonian.
    }
    
    \begin{proof}
        Let's remember that a \textbf{hamiltonian graph} is a graph that contains a \textbf{hamiltonian cycle} and a hamiltonian cycle is
        a cycle that contains all the vertices.\pn
        
        Now, lets remember what is the Petersen's graph. The Petersen's graph has the following graphic representation:
        \begin{center}
            \includegraphics[width=7cm]{PetersenIsNotHamiltonian/Petersen1.png}    
        \end{center}\pn
        
        Notice that Petersen's graph is \textbf{cubic}, that is, each vertex has degree three.\pn 

        Notice that any cubic graph must have an even number of vertices, because, if we call $n$ to the number of vertices, 
        the number of edges must be $\frac{3n}{2}$, since this number is integer, $n$ must be even.\pn
        
        Now, lets construct a very useful tool to determine if a cubic graph could be hamiltonian.\pn 
        
        Let $G = (V, E)$ be any graph, a \textbf{$k$-edge-colouring} is a suprayective function $f : E \longrightarrow \{1, 2, \dots, k\}$, 
        were $\{1, 2, \dots, k\}$ is our colors set. A \textbf{proper $k$-edge-colouring} is a $k$-edge-colouring such that each edge has a 
        different color from any of its adjacent edges.\pn
               
        For example, any even cycle has a proper 2-edge-colouring, to see this, simply choose any edge and give it the first color, 
        then choose any of its two adjacent edges and give it the second, then choose the next edge and give it the first color, and 
        repeat this procedure until you reach the first edge you chose.\pn
            
        Lets suppose that $G$ is a cubic hamiltonian graph and lets call $C$ any of its hamiltonian cycles. Since $C$ contains every 
        vertex of $G$ and $G$ has an even number of vertices, then $C$ is an even cycle, so it has a proper 2-edge-colouring. Since 
        $G$ is cubic, any two edges that are not contained in $C$ must be non-adjacent, since the opposite will give us a vertex of
        degree at least four and this can not happen. Give a third color to every edge of $G$ that is not contained in $C$ and then
        we get a proper 3-edge-colouring.\pn
        
        What we have is that if a graph is cubic and hamiltonian, then it has a proper 3-edge-colouring. We are going to show that
        Petersen's graph can not be hamiltonian since it doesn't have a proper 3-edge-colouring.\pn
        
        Lets try to give a proper 3-edge-colouring to Petersen's graph. Lets look at the 5-cycle that is in the ``outside'' of 
        the graphic representation of Petersen's graph. This cycle must have three colors since it is odd, and any color must has, at most,
        two edges (since any subset of three edges has at least two adjacent edges). This says that any proper 3-edge-colouring of
        this 5-cycle has two edges of one color, two edges of a second color, and only one edge of a third color. Lets say it has
        two ``blue'' edges, two ``red'' edges and one ``green'' edge. Because of the symmetries of Petersen's graph, it doesn't matter
        how we set this colors in its ``exterior'' 5-cycle. So, lets say that the following picture represent any of these:
        \begin{center}
            \includegraphics[width=7cm]{PetersenIsNotHamiltonian/Petersen2.png}    
        \end{center}\pn
        
        Now, notice that if two of the edges of a given vertex already have a color, then the color of the third one is determined.
        Applying this, we can color as follows:
        \begin{center}
            \includegraphics[width=7cm]{PetersenIsNotHamiltonian/Petersen3.png}    
        \end{center}\pn
        
        Now, notice that if an edge is such that both of its vertices have edges with different colors, then the color of this edge
        is already determined. Lets color one edge following this.
        \begin{center}
            \includegraphics[width=7cm]{PetersenIsNotHamiltonian/Petersen4.png}    
        \end{center}\pn
        
        One of the vertices circled with orange already had a green edge, and the other one already had a red edge, that lets the edge that joins these
        vertices no choice but to be blue colored. Now, look at the edge circled with orange in the next picture:
        \begin{center}
            \includegraphics[width=7cm]{PetersenIsNotHamiltonian/Petersen5.png}    
        \end{center}\pn
        
        It also has a vertex with an already green colored edge and another one with an already red colored edge, so it must be blue also,
        but one of its vertices already had a blue colored edge too. So, our colouring for the ``exterior'' 5-cycle cannot be chosen
        that way, since it lead us to this colouring issue. But it doesn't matter what colouring we choose for this 5-cycle, all of them
        will be basically the same (up to symmetry), and therefore, Petersen's graph cannot be hamiltonian.    
    \end{proof}